\documentclass{article}
\usepackage[utf8]{inputenc}
\begin{document}

\title{95845 Project Proposal}
\author{\name Lizhi Zhao \email lizhiz@andrew.cmu.edu \\
       \AND
       \name Meghan Clark \email meghanc@andrew.cmu.edu\\
       \AND
       \name Xuliang Sun \email xuliangs@andrew.cmu.edu}
\date{March 2018}
\maketitle

\section{Proposal Details} \label{details}

\subsection{What is your proposed analysis? What are the likely outcomes?}
Our proposed analysis is to estimate the global energy consumption of a building using data on whether or not it is a holiday, building location, historical weather data for geographic area close to the building, type of building, and historical building consumption data. 
With the combination of these data sources, our team will work towards creating a model that predicts building level energy use. 


\subsection{Why is your proposed analysis important?}
The proposed analysis is important because the majority of energy inefficiencies occur during the transportation of energy. This is why there has been a recent movement for local energy production and energy grids. However for energy companies to confidently move from a national network to localized distribution, they  need the tools to better match supply with demand. The only way to accomplish this match is by improving building level energy use forecasts. 

\subsection{How will your analysis contribute to existing work? Provide references.}
The main goal of this project is select a machine learning algorithm to come up with
a model that is more robust and precise to forecast building energy consumption with
little data.

\subsection{Describe the data. Please also define Y outcome(s), U treatment, V covariates, W population as applicable.}
Four data sets are available for this project.
\item Historical Consumption: A selected time series of consumption data for over 200 buildings.
\item Building Metadata:Additional information about the included buildings.
\item Historical Weather Data: This data set contains temperature data from several stations near
each site. For each site several temperature measurements were retrieved from stations in 
a radius of 30 km if available.
\item Public Holidays: Public holidays at the sites included in the data set, which may be helpful 
for identifying days where consumption may be lower than expected.

\item Y outcomes are the energy consumption values
\item U no treatment in this project
\item V covariates are Surface(The surface area of the building),Temperature(The temperature as measured at the weather station)
\item W population are building sites being considered


\subsection{What evaluation measures are appropriate for the analysis? Which measures will you use?}
For each building and test period, the quality of the forecast will be evaluated using the Weighted Root Mean Squared Error (WRMSE) measure.

\subsection{What study design, pre-processing, and machine learning methods do you intend to use? Justify that the analysis is of appropriate size for a course project.}
Because the Timestamp given by the training data are not in proper format and the time period is also not what we can use directly and we have to pre-processing the time data. We intend to use neural network for this project because we think the dataset may not be linear separable.In our dataset, more than 200 building sites are considered and three time horizons and time steps are distinguished, so it should be a reasonable size for a course project.

\subsection{What are possible limitations of the study?}
The possible limitation may be omitting some other important features which could have big effect on the energy consumption.

\subsection{References}
https://www.drivendata.org/competitions/51/electricity-prediction-machine-learning/page/101/

\end{document}
